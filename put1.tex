\documentclass{IEEEtran}

\usepackage[dvips]{graphicx}
\usepackage{amsmath}

\title{Alternative route search algorithm for robust and balanced traffic in telecommunication network}
\begin{document}

\maketitle

\section{Introduction}

Routing policy in optical scheme must meet numerous, sometimes,
excluding requirements. The most obvious is to find the shortest
patch among available routes. However, in optical networks
many other factors must be considered, such as: existence of the protection
path, global power consumption, power levels, physical 
distance as well as a logical distance measured in 
number of routers and regenerators between egress and ingress \cite{load1,load2}.

The Dijkstra's shortest path algorithm, however widely used and 
optimal in terms of single path can be not the optimal choice 
for the whole network.  For a real telecommunication infrastructure, with multiple
tunnels held in the same network, a number of problem arises \cite{path1}:
\begin{itemize}
\item Reserving resources for the optimal path can block further 
searches.
\item The traffic is not equalized between links. Malfunction of 
highly loaded links can cause non-restorable data loss.
\item Contextless resource reservation makes the first 
paths provisioned in the network privileged compared to the others.
This results in nondeterminism in resource allocation.


\end{itemize}


Searching for optimal 
The possible issues are inequal resources 






\section{Naive walker algorithm}

The proposed routing algorithm is based on the following 
heuristics. Given a graph shown in Figure \ref{f1}.
Suppose one want to reach node $n5$ starting from
$n1$. Also assume, that in $n5$ there is a big tower, so 
that the direction one should walk is known.
Thus, among all possible intermediate nodes:
$n3$, $n4$ and $n2$ the last one seems the best choice.

Reaching $n2$, however, is a blind route and a person 
must move back and chose either $n3$ or $n4$. 
Thus, there is no guarantee that the choice of 
path in the direction closest to the destination
is the optimal chice, but it is more or less how
a human would behave.

\begin{figure}
  \centering
  \includegraphics[width=.3\textwidth]{xx.eps}
  \caption{Example graph for naive path search}
  \label{f1}
\end{figure}

The mathematical formulation of this fact is as follows:
Let $n_i$ is a vector pointing from the origin to the
$i$th node of the graph. Then, standing in $n_1$,
and knowing the destination node $n_E$, as the next
chop, among incident ones $n_i$, one should choose the
one for which the dot product  (denoted as $\circ$), $l_i$:  
\begin{equation}
  \label{eq:2}
  l_i=\frac{(n_E-n_1) \circ (n_i-n_1 }{||(n_E-n_1||\, ||n_i-n_1|| }
\end{equation}
is maximized. Provided no blind node is reached, the procedure is repeated until the 
target is reached. 

This very simple algorithm has a number of drawbacks: It is possible to reach
a termination node (as presented in Fig \ref{1}), the route does not have to be optimal.
 However, a second part of the described technique is preparation of graph 
in a manner which minimizes the risk of such incidents. 
Thus, the greedy search needs a complementary graph embedding algorithm,
which creates ``regular'' graph.

\section{Spectral graph embedding}

Graph embedding, or, in other words plotting a graph on a plane 
(if possible) is a demanding task. An intuitive approach is based
on treating edges as springs and by finding the stable 
configuration in terms of mechanical system. 
Another approach is spectral embedding, where the
coordinates where nodes are placed are based on eigenvectors
of the Laplacian matrix of a graph \cite{lap}.


\begin{figure}[!ht]
  \centering
  \includegraphics[width=.3\textwidth]{xx2.eps}
  \caption{Example graph for naive path search}
  \label{f2}
\end{figure}

\begin{equation}
  \label{laplacian}
\mathbf{L}=\left[
\begin{matrix}
-3 & 1 & 1 & 1 & 0 & 0 \\
 1 &-1 & 0 & 0 & 0 & 0 \\
 1 & 0 &-2 & 0 & 1 & 0 \\
 1 & 0 & 0 &-2 & 0 & 1 \\
 0 & 0 & 1 & 0 &-2 & 1 \\
 0 & 0 & 0 & 1 & 1 & -2
\end{matrix}
\right]  
\end{equation}
% [ -3   1   1   1   0   0 ;  1  -1   0   0   0   0 ;  1   0  -2   0   1   0 ;  1   0   0  -2   0   1 ;  0   0   1   0  -2   1 ;  0   0   0   1   1   -2; ]


%  -4.0825e-01   2.3661e-01  -1.0463e-16   4.0825e-01  -2.2204e-16   7.8146e-01
%  -4.0825e-01   7.8146e-01   3.9194e-17  -4.0825e-01   1.9429e-16  -2.3661e-01
%  -4.0825e-01  -1.1830e-01  -6.0150e-01   4.0825e-01   3.7175e-01  -3.9073e-01
%  -4.0825e-01  -1.1830e-01   6.0150e-01   4.0825e-01  -3.7175e-01  -3.9073e-01
%  -4.0825e-01  -3.9073e-01  -3.7175e-01  -4.0825e-01  -6.0150e-01   1.1830e-01
%  -4.0825e-01  -3.9073e-01   3.7175e-01  -4.0825e-01   6.0150e-01   1.1830e-01

Spectral embedding, despite some issues described later in this section, possesses
several important properties:
\begin{itemize}
\item Strongly connected nodes (i. e. the pairs for which exists more different 
paths) are drawn closer to each other. 
\item Nodes connected with edge of higher weight are closer.
\item Regular graphs looks regular. 
\end{itemize}
The idea behind using an eigenvectors as nodes' coordinates comes from 
Laplace matrix, representing $2$nd order differential operator.



\begin{thebibliography}{9}
  \bibitem{main}
J. de Santi, A. Drummond, N. de Fonesca, X. Chen, A. Jukan. 
{\emph Leveraging Multipath Routing and Traffic Grooming for and Efficient Load Balancing in Optical Networks}
IEEE Trans on Optical Networks and Systems, pp 2989--2993, 2012.


\bibitem{load1} 
A. Rahman, N. M. Sheikh, {\emph Modified Bidirectional Reservation on Burst drop with Dynamic Load Balance in Optical Burst Switching},
11th Int Conference on Telecommunications - ConTel 2011, Graz Austria.

\bibitem{load2}
J. Zhang, S. Wang, K. Zhu,L. Sone, D. Datta, Y. Kim, B. Mukherjee,
{\emph Optimized Routing for Fault Management in Optical Burst-Switched WDM Networks}
IEEE Journal of Selected Areas in Communication, Vol. 25, No 6, Aug 2007, pp 111--120

\bibitem{path1}
M. Chamania, A. Jukan.
A Survey of Inter-Domain Peering and Provisioning Solutions for the Next Generation Optical Networks,
IEEE Communication Surveys and Tutorials, Vol 11, No 1, 2009, pp. 33--51.


\end{thebibliography}


\end{document}
